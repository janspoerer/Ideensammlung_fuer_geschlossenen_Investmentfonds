\documentclass{article}
\usepackage[utf8]{inputenc}

\title{Geschlossener Fonds}
\author{Jan Spoerer}
\date{December 2019}

\begin{document}

\maketitle

\section{Ziel dieses Dokuments}

Ich plane, mittelfristig einen geschlossenen Wertpapier zu eröffnen und selbst zu verwalten. Dies hat zwei Gründe: 1) Umschichtungen von Wertpapiervermögen natürlicher Personen werden in Deutschland steuerlich benachteiligt, 2) dieses Konstrukt ermöglicht es mir, in meinem Bekanntenkreis auch anderen Menschen einfachen Zugang zu meinen Entscheidungen zu geben. Dieses Dokument soll Auskunft über meine Pläne geben und auch als Maßstab dienen. Denn wenn ich diese Idee umsetze, kann der Leser dieses Dokument heranziehen und selbst prüfen, ob meine hier gemachten Aussagen sich als sinnvoll erwiesen haben und ob ich meine eigenen Ziele erreichen konnte.

Zu Punkt 1): Gelegentlich schichte ich mein Wertpapierportfolio um. Das heißt, dass ich Wertpapiere verkaufe und davon neue Wertpapiere kaufe. Leider werden beim Verkauf von Wertpapieren bereits Kapitalertragssteuern fällig, auch wenn man das Geld direkt wieder in Wertpapiere investiert. Da ich einen sehr langen Anlagezeitraum habe, machen diese vorzeitigen Steuerzahlungen einen Teil der möglichen Zinseszinseffekte zunichte, die ich gerne nutzen möchte. Wenn man Geld in einem Fonds gebündelt hat, sind diese steuerlichen Probleme nicht vorhanden.

Zu Punkt 2): Wie du wahrscheinlich weißt, gebe ich bereits heute auf freundschaftlicher Basis Auskunft über mögliche Investments und verwalte für einige Familienmitglieder auch aktiv Depots. Ich habe bemerkt, dass die Nachfrage nach mehr Sicherheit und mehr Rendite sehr hoch ist und dass viele Leute in meinem Umfeld meine Ratschläge schätzen. Wenige Leute befolgen diese Ratschläge jedoch. Ich vermute, dass das auch daran liegt, dass Wertpapierinvestments zu viel anfängliche Überwindung erfordert. Die meisten Leute interessieren sich nicht wirklich für das Thema, sondern wollen ihr Geld lediglich sicher und profitabel anlegen. Dies ist jedoch oft nicht genug, um erfolgreich Geld anzulegen. Dieser Fonds ermöglich es Bekannten, diese Überwindung nicht haben zu müssen.

\section{Anlagephilosophie}

Der Fonds wird auf meine eigene Anlagephilosophie ausgerichtet und entspricht meinem Risikoprofil, meinen Liquiditätsbedürftnissen und meinem Anlagehorizont. Kurz gesagt fordere ich von einem Portfolio, dass es langfristig gesehen einen maximalen Erwartungswert aufweist. Zwischenzeitige Schwankungen sind mir egal. Ich brauche das Geld aus dem Fonds nicht und denke eher an nachfolgende Generationen. Zwischendurch soll das Portfolio zwar attraktive Ausschüttungen ermöglichen, die zum Beispiel für die Altersvorsorge genutzt werden können, jedoch möchte ich nie an die Substanz gehen.

\section{Messbarkeit, Leistungsüberwachung, Governance}

Mir ist wichtig, dass die Ziele des Fonds für jeden Investor einfach verständlich sind. Im Nachhinein soll mit wenigen Kennzahlen klar erkennbar sein, ob ich meine Ziele erreicht habe oder nicht. Erfolg und Misserfolg sollen sich nicht durch Erklärungen relativieren lassen können.

Deshalb werde ich, wie jeder andere Fonds auch, die Returns des Fonds kontinuierlich messen. Mein Ziel ist, eine Rendite nach allen Gebühren zu erzielen, die der theoretischen Rendite des MSCI World entspricht. Dies ist sehr ambitioniert, fast kein Fondsmanager hat es in der Vergangenheit geschafft, übliche Benchmarks wie den MSCI World zu schlagen. Dazu kommt mein Anspruch, dies selbst nach Gebühren zu schaffen.

Ich nehme keine Volatilitätskennzahlen in die Performancemessung auf. Meiner Meinung nach spielt Volatilität für sehr langfristig ausgerichtete Anleger keine große Rolle. In der Literatur wird Volatilität auch oft schlicht Risiko genannt. Je nach Anlagehorizont ist diese Bezeichnung jedoch nicht hilfreich und hält Anleger nur davon ab, in langfristig überlegene Anlagen wie Aktien zu investieren.

Sollte es zu einer Wirtschaftskrise kommen, wird dies auch den MSCI World treffen. Meine Performance steht also immer in Relation zum Gesamtmarkt. Ich behaupte nicht, völlig unabhängig vom Gesamtmarkt agieren zu können. Jeder Investor ist fast völlig den Bewegungen am breiten Markt ausgesetzt. Jedoch können bereits kleine Vorteile gegenüber dem Gesamtmarkt zu enormen langfristigen Vermögenszuwächsen führen. Diesen kleinen Vorteil möchte ich erreichen und ich denke, dass die relative Messung meiner Leistung somit am besten geeignet ist.

Dies hätte in der Vergangenheit geheißen, dass Returns von durchschnittlich etwa 6\% jährlich erzielt worden wären. In Deutschland fallen auf Kapitalerträge zusätzlich die Kapitalertragssteuer sowie der Solidaritätszuschlag an.

Ich habe den MSCI World gewählt, da er eine gute Übersicht über die weltweite Performance von Aktien gibt.

\section{Investitionsbedingungen}

Ich werde stets mindestens 30\% des Gesamtvermögens selbst halten. Vor jeder neuen Einzahlung wird geprüft, ob ich genug Kapital im Fonds investiert habe, um diese Quote einzuhalten. Dies stellt sicher, dass ich auch selbst genügend an einer guten Performance des Fonds interessiert bin. Investoren können sich sicher sein, dass die Entscheidungen des Fonds vollkommen nach bestem Wissen und Gewissen durchgeführt werden. Somit ist das Gesamtkapital durch mein selbst investiertes Kapital begrenzt und der Fonds nur langsam und organisch über die Zeit wachsen.

Dies kann dazu führen, dass Investoren warten müssen, bis ein Investment ausgeführt werden kann. Denn voraussichtlich wird es dazu kommen, dass ich zwischendurch immer wieder einige Monate Geld ansparen und in den Fonds investieren muss, bevor wieder neues externes Geld in den Fonds fließen kann. Hierfür ist eine Warteliste geplant, 




\section{Gleichbehandlung verschiedener Investoren}

Alle Investoren werden gleich behandelt, zahlen also dieselben Gebühren. Außerdem sind ausdrücklich keinerlei Freundschaftsangebote geplant, die werbenden Investoren ermöglichen, einen Bonus für Neuinvestoren zu bekommen. Dies hat zwei Gründe: Zum einen sollen Neuinvestoren sich durch den Fonds an sich angesprochen sehen und nicht durch Bekannte zu einem Investment gedrängt werden. Zweitens erwarte ich nicht, dass sich zu wenige Investoren finden werden lassen. Und selbst falls dies der Fall sein sollte, sollte ich mir eher über meinen Investmentansatz Gedanken machen und nicht über Marketingmaßnahmen wie Freunde-werben-Freunde-Programme. 
\section{Rechtliches}

Diesem Dokument liegen keine tatsächlichen Investmentmöglichkeiten zugrunde. Ich möchte lediglich über meine Pläne informieren, einen Fonds einzurichten. Sobald dies geschieht, wird ein offizieller Verkaufsprospekt erstellt, der rechtlichen Anforderungen entspricht.

\section{Anhang: Einschätzungen zu bestimmten Anlageformen}

\subsection{Übersicht}

Im Folgenden stelle ich meine Einschätzung zu den gängigsten Anlageformen kurz vor. Von den genannten Anlageformen nutze ich nur: Aktien, Anleihen, Immobilien, P2P und Kryptowährungen. Von anderen Anlageformen rate ich aus den unten genannten Gründen eher ab.

\subsection{Mögliche Anlageformen (``Asset Universe'')}

\subsubsection{Sachwerte}

\paragraph{Aktien}

\paragraph{Immobilien}

Immobilien 

Eine der Besonderheiten von Immobilien ist, dass Immobilien einfach mit Fremdkapital, also Bankdarlehen, gekauft werden können. Das führt dazu, dass Investoren bei geringen Mietrenditen trotzdem noch hohe Eigenkapitalrenditen erzielen können, wenn alles nach Plan läuft. Dies führt jedoch auch dazu, dass viel Kapital in Immobilienmärkte fließt und Mietrenditen somit oft geschmältert werden und sich ein Investment ohne Fremdkapital nicht immer lohnt. 

\paragraph{Gold}

Gold ist aktuell das einzige Metall, wessen Preis im Wesentlichen durch nichtindustrielle Nachfrage gestützt wird. Deshalb ist Gold die einzige mir bekannte Anlageform (außer bestimmter Derivate, Zertifikate und Anleihen), die eine negative Korrelation mit dem Aktienmarkt aufweist und somit als Risikoausgleich dienen kann.

Leider hat der Kauf von Gold jedoch keinerlei produktive Leistung zur Folge. Anders als der Kauf von Aktien, Immobilien oder Anleihen kann mit dem Kauf von Gold keine sinnvolle Kapitalallokation in gewinnbringende Projekte erfolgen. Sprich: Es gibt keinerlei Dividenden, Mieterträge oder Zinsen. Die Kaufkraft von Gold ist in der Regel sehr stabil. Im alten Rom soll eine Toga eine Goldunze wert gewesen sein und auch heute noch kann man sich einen hochwertigen Anzug von einer Unze Gold kaufen. Das Problem dabei ist jedoch, dass man mit einem Investment in Gold in den letzten 2000 Jahren somit keinerlei reale Rendite hätte erzielen können.

Für einen langfristigen Investor sollte Gold deshalb mit großer Vorsicht gehandhabt werden, ich werde maximal 5\% des Fondsvermögens in Gold investieren.

\paragraph{Andere Rohstoffe}



\subsubsection{Geldwerte}

\paragraph{Anleihen}

Anleihen werden seit vielen Jahrzehnten als elementarer Bestandteil eines Portfolios angesehen. Staatsanleihen von bonitätsstarken Ländern können sogar negativ mit Aktienmärkten korreliert sein und damit eine ähnliche Absicherungsfunktion wie Gold übernehmen.

Im ``All Weather Portfolio'' von Ray Dalio spielen Anleihen eine enorm wichtige Rolle. Ray Dalio ist der Gründer einer der bedeutensten Hedge Fonds weltweit und einer der größten Ikonen der Branche. Dalio empfielt, dass 55\% eines Portfolios aus Anleihen verschiedener Emittenten (Staaten und Unternehmen) und Durationen (Fachbegriff für Restlaufzeiten) bestehen sollen.

Zusammengefasst halte ich Anleihen im heutigen Marktumfeld für Investments mit viel Risiko ohne Ertrag.

Leider ist der Markt für Staatsanleihen durch weltweite Quantitative Easing-Programme von Notenbanken und durch regulatorische Beschränkungen von Lebensversicherern und Pensionsfonds künstlich aufgeheizt. Renditen sind derart niedrig, dass ein Investment in Anleihen heute enorme Risiken birgt. Denn Anleihen mit längeren Laufzeiten können stark fallen, wenn sich die Zinsen wieder etwas erhöhen. Investoren, die im heutigen Niedrigzinsumfeld einsteigen, müssen schon darauf hoffen, dass die Zinsen für bonitätsstarke Länder wie der Bundesrepublik unter 0\% fallen. Andere Szenarien führen langfristig zu Risiken und vorprogrammiert geringen Erträgen.

Der Markt für Unternehmensanleihen hat sich analog zum Markt für Staatsanleihen verhalten und ist meiner Meinung nach ähnlich überhitzt.



\paragraph{Euro und Fremdwährungen}

Auch Bargeld oder Buchgeld zu halten ist nicht risikolos. Auch im Bereich Geldanlage gilt ``Man kann nicht nicht kommunizieren''. Wer sein Geld als Bargeld hält, geht Inflationsrisiken ein. Sollte der Eurokurs gegenüber anderen Währungen absacken, kosten importierte Produkte mehr. Sollte die Geldmenge in der Eurozone stark ausgeweitet werden, könnten auch heimische Produkte teurer werden.

Fremdwährungen wie der US-Dollar bringen natürlich ebenfalls Risiken. Ich werde nicht aktiv mit Fremdwährungen (z.B.\ in Form von Swaps) spekulieren, jedoch sind Wertpapiere im Portfolio, die in Fremdwährungen notiert sind.

\paragraph{P2P-Kredite}



\subsubsection{Sonstiges}

\paragraph{Wandelanleihen} 

Wandelanleihen sind ein Hybrid aus Aktienoptionen (Abschnitt \ref{optionsscheine}) und Anleihen (Abschnitt \ref{anleihen}). Somit haben Wandelanleihen sowohl eine Sachwert- als auch eine Geldwertkomponente.

\paragraph{Zertifikate}

\paragraph{Optionsscheine} \label{optionsscheine}

\paragraph{Kryptowährungen und weitere Token}

Kryptowährungen und Blockchain-Token sind ein weites Feld. Kurz gesagt: Blockchain ist an sich gar keine Anlageklasse. Zumindest würde ich es nicht so einstufen. Bitcoin ist eine Sondererscheinung im Bereich Blockchain, weil man dort tatsächlich in die Blockchain investiert. In der Regel ist Blockchain aber etwas anderes: Eine Technologie, die bestimmte Arbeitsabläufe wie zum Beispiel das Zahlungssystem verbessern soll.

Zusätzlich gibt es noch Beschränkungen und Probleme bei Investments in Kryptowährungen. Ich werde, wenn sich diese Probleme mit angemessenem Aufwand bewältigen lassen, bis zu 5\% des verwalteten Vermögens in Bitcoin investieren.

\end{document}
